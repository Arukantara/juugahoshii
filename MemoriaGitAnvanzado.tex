\documentclass{article}
\usepackage[utf8]{inputenc}

\title{Memoria GIT avanzado}
\author{Alumno 1: Enrique Arauz Morales\\
        Alumno 2: José Alcantara Muñoz\\
        Alumno 3: Diego Godoy Poce\\
        Alumno 4: Daniel Guerrero Molina\\
        Becario: Francisco Madueño Chulián}
\date{May 2014}

\begin{document}

\maketitle

\section{Preparación del entorno}
Empezamos la memoria
\section{No queremos tantos commits}
\section{Creando la rama de tu equipo}
\section{Trabajando con un repositorio remoto}
\section{Clave pública SSH}
\section{Generando números}
\section{Merge sin conflictos}
\section{Merge}
\section{Alumno1 vs alumno3 (conflictos)}
\section{Ficheros a ignorar (alumnos 2 y 4)}
\section{Fetch + merge (alumnos 1 y 3)}
\section{Alumno2 vs alumno4 (conflictos)}
\section{Dividiendo commits}
\section{Rebase manejo de ramas}
\section{¿¡Quién hizo eso!?}


\end{document}
